\chapter*{Conclusion générale}
\addcontentsline{toc}{chapter}{Conclusion générale}
\markboth{Conclusion générale}{}

\par À travers ce projet réalisé au sein d'Avaxia Group, nous avons pu constater la nécessité de disposer de systèmes performants pour gérer, 
stocker et analyser les données à grande échelle. La plateforme analytique conçue représente une avancée significative dans cette direction, 
offrant à l'entreprise les outils nécessaires pour extraire des informations pertinentes et exploiter son potentiel de manière optimale.

\par En retraçant les différentes étapes du projet à travers les chapitres dédiés au cadre, à l'analyse des besoins, à l'architecture et
 à la réalisation, nous avons mis en évidence l'importance de chaque phase dans la réussite de l'ensemble du projet. De la définition 
 des objectifs à la conception détaillée en passant par le choix des technologies et le développement des micro-services, chaque étape 
 a contribué à la création d'une solution cohérente et fonctionnelle.
 
 \par Ce projet nous a permis de mettre en pratique nos compétences techniques et méthodologiques, tout en nous confrontant à des défis
  réels du monde professionnel. Travailler au sein d'Avaxia Group nous a également donné l'opportunité de collaborer avec des professionnels
   expérimentés et d'évoluer dans un environnement stimulant et enrichissant.

   \par Après avoir mené à bien ce projet, plusieurs perspectives d'amélioration et d'extension se dégagent,
 ouvrant la voie à des nouvelles opportunités et à une optimisation continue de la plateforme analytique conçue. Telque, l'intégration d'un assistant AI 
 pour l'optimisation des requêtes SQL représente une avancée majeure dans l'amélioration de l'expérience utilisateur. 
 
 \par De plus, l'ajout d'un portail utilisateur spécifique aux besoins définis permettra une expérience plus intuitive et personnalisée. 
 Les utilisateurs pourront ainsi accéder facilement aux fonctionnalités pertinentes et adaptées à leurs responsabilités, favorisant ainsi 
 l'adoption et l'utilisation de la plateforme.
 
  \par En outre, l'intégration de l'authentification avec le Single Sign-On (SSO) renforcera la sécurité et simplifiera la gestion des accès à la plateforme.
   Cette fonctionnalité permettra aux utilisateurs de se connecter en toute sécurité en utilisant leurs compte Snowflake ou Azure existants directement, améliorant 
   ainsi la sécurité globale de la solution.

 \par Enfin, le déploiement de la solution à grande échelle est essentiel pour répondre aux besoins croissants de l'entreprise. Cette étape comprendra non seulement 
 la mise en production de la plateforme dans un environnement opérationnel, mais aussi son intégration avec d'autres systèmes et outils utilisés par l'entreprise, 
 garantissant ainsi une performance optimale même en cas de forte demande.

 \par En conclusion, ce projet a permis de poser les bases d'une plateforme analytique robuste et évolutive pour Avaxia Group. Les perspectives envisagées offrent 
 des améliorations significatives qui renforceront encore d'avantage la capacité de l'entreprise à gérer et à exploiter ses données de manière optimale.
  Nous sommes convaincus que ces évolutions contribueront à maintenir Avaxia Group à la pointe de l'innovation technologique et à soutenir sa croissance future.
