\chapter*{Introduction générale}
\addcontentsline{toc}{chapter}{Introduction générale} % to include the introduction to the table of content
\markboth{Introduction générale}{} %To redefine the section page head

%Exemple d'utilisation de la bibliographie utilisée \cite{webArticle2}. Le style utilisé est IEEE \cite{webArticle1}.\\
\par Dans un monde où les données jouent un rôle crucial dans la prise de décision stratégique, 
la gestion efficace des données est devenue un impératif pour la compétitivité des entreprises.
De plus avec l'avènement des plateformes de data warehousing cloud telles que Snowflake, une plateforme de 
données en nuage reconnue pour sa flexibilité et ses performances,les entreprises nécessitent désormais d'outils puissants pour gérer,
stocker et analyser leurs données à grande échelle. Ces systèmes de gestion de données performants permettent d'extraire 
des informations pertinentes et d'exploiter leur potentiel de manière optimale, ouvrant ainsi de nouvelles perspectives de croissance et d'innovation.
\par Cependant, exploiter les capacités de Snowflake requiert une surveillance constante et une optimisation 
proactive des opérations effectuées sur cette plateforme. C'est dans ce contexte que s'intègre notre sujet de fin d'études 
pour l'obtention du diplôme national d'ingénieur, effectué au sein de l'entreprise Avaxia Group, qui nous a accueilli dans
un environnement créatif et sérieux et qui nous a confiée de concevoir et d'implémenter une platforme analytique pour la 
visualisation des méta données de Snowflake son platforme d'entrepôt de données principal.

\par Pour retracer l'acheminement chronologique de notre travail, le présent rapport a été subdivisé en quatres chapitres 
principaux, chacun abordant une étape essentielle du projet, depuis la définition du cadre et des besoins jusqu'à la réalisation
 technique et la mise en œuvre de la solution.
 \par Le premier chapitre, \textbf{Cadre du projet}, pose les bases en introduisant le contexte général du projet, l'organisme d'accueil, 
 et en analysant les enjeux auxquels nous faisons face. Il identifie également la problématique,propose une analyse critique
  de l'existant et définit les objectifs et solutions envisagées. Enfin, un choix méthodologique est justifié pour assurer une 
  gestion de projet efficace.
\par Le deuxième chapitre, \textbf{Analyse et spécification des besoins}, se concentre sur la capture des besoins fonctionnels 
et non fonctionnels. Nous y détaillons les concepts clés liés à Snowflake, les solutions de monitoring et les techniques
d'analyse de données. Un diagramme de cas d'utilisation global est présenté pour visualiser les interactions système-utilisateur,
accompagné d'un flux de travail détaillant le processus de l'extraction à la visualisation des données.
Le backlog du produit est également élaboré pour structurer et prioriser les tâches à accomplir.
\par Dans le troisième chapitre, \textbf{Architecture et conception}, nous développons l'architecture logicielle et matérielle du système
 proposé. L'étude architecturale aborde à la fois les aspects logiques et physiques,
 tandis que l'étude conceptuelle se focalise sur la conception globale et détaillée de la solution, garantissant une structure robuste et scalable.\\
 \par Enfin, Le quatrième et dernier chapitre, \textbf{Réalisation}, décrit le choix des technologies, les frameworks et outils utilisés
  pour le développement, ainsi que les étapes de réalisation des différents micro-services composant notre système de monitoring.
   Chaque micro-service est détaillé, illustrant la manière dont ils s'intègrent pour former une solution cohérente et efficace.
   \par Nous clôturons notre rapport par une conclusion générale qui résume les réalisations essentielles de notre travail et propose des éventuelles perspectives.